\documentclass[11pt,twocolumn]{book}
\usepackage{parskip}
\usepackage{titlesec}
\usepackage{mdframed}

\setlength{\headsep}{0cm}
\titleformat{\chapter}[display]% NEW
    {\fontfamily{ptm}\large\bfseries\centering}{\sc\chaptertitlename\ \thechapter}{5pt}{\large}% NEW
\titlespacing*{\chapter}{0pt}{-50pt}{20pt}% NEW

\begin{document}
\pagestyle{headings}

\chapter{INTRODUCTION TO DOMINION RULES}

Dominion Rules (DR) is a rules system for historical or fantasy roleplaying. These kinds of roleplaying games take place in mediaeval or quasi-mediaeval settings such as the Sherwood Forest of Robin Hood, King Arthur’s Britain or the Middle Earth of JRR Tolkien’s novels. While DR was created with these sorts of settings in mind, it can be readily adapted to other settings. DR is designed to be easy to develop and expand.

WHAT YOU NEED TO PLAY

To play DR, you need the following:
\begin{itemize}
\item  These rules
\item  Some paper and pencils
\item  At least  two, but preferably three or more players, one of whom will serve as Games Master
\item  At  least one twelve-sided  die. It is  best  to have one for every player. 
\end{itemize}

You can get a twelve-sided die from the local hobby shop or order one from games shops on the internet.

TERMINOLOGY

DR uses some key terms you need to be familiar with before reading on.

\textbf{Games Masters}

Roleplaying games usually have one referee and several players. DR calls the referee the Games Master, or GM. The GM’s role is to guide players through their adventures, describing the dominions, peoples and creatures they encounter and applying the rules. One player will be the GM and the others will play characters in the dominion.

\textbf{Characters}

The personalities the players control in the game are called player characters (PCs), or simply charac- ters. But the term character may also refer to char- acters controlled by the GM; these are known more
specifically as Games Mas- ter characters (GMCs). In DR, most every rule that applies to PCs applies to GMCs, too.

\textbf{Rules and Dominions}

DR makes a distinction be- tween rules and dominions.
A twelve-sided die (d12)
The rules are simply the rules of the game, as de- scribed here.
A dominion is the world your characters inhabit. It’s what is known in some RPGs as the campaign set- ting. GMs can create their own dominions or use dominions created by others.
These rules do not describe any particular domin- ion. They are intended to apply to any dominion, unless your GM says otherwise. For example, if the dominion you are playing in is strictly historical, such as Europe during the Hundred Years’ War, your GM may tell you that the ordinary DR rules about witchcraft and priestcraft do not apply. In other dominions, the GM may allow certain char- acter races but not others. And so on.

\textbf{Stats and Rolls}

DR makes an important distinction between a stat and a roll. Stat is short for statistic: a numerical fig- ure used to measure a character’s ability in some way. A roll is the number produced by rolling the twelve-sided die.
Usually in DR, characters must roll less than or equal to their Skill stat in order to succeed at the task they are attempting. The result of the roll will then be used in play. Sometimes, however, it is the stat rather than the roll that is important. Be sure not to confuse stats with rolls.

HOW TO USE DR

How you use Dominion Rules is up to you.
If you want to play in a traditional fantasy roleplay- ing setting, Dominion Rules gives you all the tools you need to do so, including rules for spellcasting, priestcraft, enchanted items and common fantasy character races. DR also gives you the framework you need to adapt the system to your own needs, for instance by creating new skills, creatures, and spells.

If you want to play in a strictly historical setting, all you need to do is simply disregard those parts of DR that have no historical basis. The rules on non- human character races, enchanted items, priestcraft and magic, and fantasy creatures can be ignored. DR is designed to be modular. It’s easy to drop cer- tain elements without affecting the balance of the rest of the game.

Another possibility is to use DR to play in a setting somewhere in between strict history and high fan- tasy. For example, to play in an Arthurian setting you might decide to use some or all of the standard DR magic rules, but none of the rules for fantasy creatures and character races.

ROLLS OF TWELVE

In DR, a roll of 12 on the twelve-sided die is always a failure. It’s the worst roll you can get. Whatever you are trying to do, you fail when you roll a 12.

EXAMPLE BOXES

Throughout these rules you’ll find illustrations of the various rules explained here.

\begin{mdframed}
EXAMPLE

Here is an example of an example box.
\end{mdframed}

CAPITALIZED TERMS

Many of the terms used in DR are written with the first letter capitalized. This is to distinguish between the technical DR sense of the word and its every- day meaning. For instance, the phrase, “Gunther Blocked” means Gunther used his Block Skill somehow, while the phrase, “Gunther blocked” means he got in someone’s way somehow, but did not use his Block Skill. When you see a capitalized word which does not normally take a capital, you’ll know the word is used in its DR sense.

DOMINIONRULES.ORG

Dominion Rules is distributed online from http://dominionrules.org. Apart from the game it- self, you’ll find discussion forums, DR news and updates, and more. Be sure to visit the site regu- larly.

DOMINION GAMES

Dominion Rules was created in the late 1990s by Dominion Games, which distributed parts of the game for free and sold other parts online. The peo- ple behind Dominion Games are now offering all of DR to the public for free.

Meanwhile, a cybersquatter has taken over our former web site, www.dominiongames.com, so DR is being distributed from a new site, http://dominionrules.org.

To contact the creators, e-mail us at dominionrules@gmail.com.

\chapter{ATTRIBUTES AND COMPOSITES}

Everyone is born with innate talents and qualities. Some people are particularly strong. Others are no- tably intelligent. Still others just seem lucky. DR calls these innate traits Attributes. Attributes deter- mine a character’s starting-point for Skill develop- ment; they answer the question, How easy is it for my character to learn to do new things?

The actions of Combat, Priestcraft and Witchcraft are central to the DR system. These actions are based on combinations of Attributes called Com- posites. Along with explaining Attributes, this chap- ter also explains Composites and the related con- cept of Favourable Rounding.

THE SIX ATTRIBUTES

Every character in the DR system possesses six At- tributes that define his or her basic qualities. They are: Vigour, Agility, Stamina, Intuition, Intellect, and Luck.

For ordinary humanoids, these Attributes are meas- ured on a scale that tends to range from 1 to 4. For non-humanoid creatures, the scale can be much larger: from 0 to 12 or even beyond. These meas- urements are called Attribute stats. Players deter- mine their characters’ Attribute stats when they create their characters.

A character’s Attribute stat never changes: it cannot be raised or lowered.

ATTRIBUTES AND SKILLS

Characters in DR possess Skills that derive from their Attributes. Skills are simply things your char- acter knows how to do. For instance, a character’s ability to read is represented by his Literacy Skill, which derives from his Intellect Attribute; a charac- ter’s ability to swim is measured by her Swimming Skill, which derives from her Vigour Attribute; and so on.

Like Attributes, Skills are measured in terms of stats. But unlike Attribute stats, Skill stats increase over time. This is known as Advancement. Note, how- ever, that a roll of 12 on a twelve-sided die always fails—even if your character’s Skill stat is 18. See chapter two, Skills.

When you create your character, her Skill stats start out equal to the Attribute stat from which the Skills derive. You then improve your character’s Skill stats with Advancement Points.

\begin{mdframed}
EXAMPLE

You have just created a new character with the following Attribute stats:

VIG: 2 AGI: 1 STA: 3 INTU: 3 INTE: 1 LUCK: 2

Until you improve your Skill stats with Advancement Points, all your character’s Vigour Skills start at 2. All your character’s Agility Skills start at 1. All Stamina Skills start at 3. And so on for all six Attributes.
\end{mdframed}

Though you can use Advancement Points to improve your character’s Skill stats, his Attribute stats will never change.

Each Attribute is explained individually below. Each Attribute’s related Skills are described in chapter two, Skills.

THE VIGOUR ATTRIBUTE

Vigour is a measure of a character’s inherent physical ability. Vigour is an expression of how big, strong, and powerful your character is naturally. Note, however, that even characters that are naturally inclined to be weak (those with a Vigour of 1) can advance their Vigour Skills. It just takes them longer.

THE AGILITY ATTRIBUTE

Agility is the measure of how dexterous and nimble a character is. Agility is an important quality for a good warrior, but is also greatly valued by spies, thieves, and other sneaks.

THE STAMINA ATTRIBUTE

Stamina is the measurement of a character’s fitness, endurance, and general health. Of course, a sickly person can take measures to improve his health and fitness (by improving Stamina Skills), but the Stamina Attribute stays the same, for health is to some extent predetermined. For example, a very fit person may nonetheless be prone to colds, or may fall victim to a hereditary disease. There are only a few Stamina Skills, but they are very important to game play.

THE INTUITION ATTRIBUTE

The Intuition Attribute is the measure of a character’s intuition. Intuition is the ability to perceive or know things without conscious use of reasoning. It is in some ways the opposite of the Intellect Attribute: it is the power to know without reasoning, whereas Intellect is the power to know by reasoning. Religious impulses are closely tied to irrational ways of knowing. Intuition is therefore important for Priests.

THE INTELLECT ATTRIBUTE

Intellect covers all facets of a character’s intelligence: reasoning, memory, wit, understanding, perception, and so on. There is an endless number of possible Intellect Skills. Most anything that can be known can be made into an Intellect Skill.

THE LUCK ATTRIBUTE

Some people are just lucky. Others seem cursed. The Luck Attribute is a measure of how kindly fate is inclined to treat a character. There are no Luck Skills, for Luck is the opposite of Skill. Yet even without Skills, Luck is an extremely useful Attribute, for Luck governs Lucky Breaks.

Lucky Breaks are a way for characters to get out of a jam. A Lucky Break gives your character a Lucky Break bonus in any Skill or Skills he makes a Skill roll for that round. The amount of the Lucky Break bonus is equal to the character’s Luck stat. Lucky Breaks can be lifesavers—literally!

You can’t just get a Lucky Break any time you like. To get one, you must have a spare Advancement Point to spend on it. If you don’t have an AP in re- serve, you cannot get a Lucky Break. See chapter three, Advancement.

You can declare a Lucky Break at any time in a round; you do not have to declare it the Strategy Stage (explained in chapter five, Order of Play). 

But keep in mind that a Lucky Break only lasts for one round. That means that if you declare it in the Resolution Stage (the last Stage of a round), you will only get the benefit of it during that Stage and until the end of the round. But if you declare it at the start of the Timing Stage (the first Stage of a round), you’ll get Lucky Break bonuses to everything your character does in that Stage and the other three Stages of the round.

\begin{mdframed}
EXAMPLE

Dame Beatrice Rideout failed her Strike roll this round. And now she has been hit with a Strike roll of 7! If she doesn’t make a good Withstand Injury roll, she’ll be in big trouble. She decides this would be a good time to get a Lucky Break!

Dame Beatrice (WithIn 8) spends 1 AP on her Lucky Break. She has a Luck stat of 3, so she gets a +3 Lucky Break bonus. Now she can roll 11 or less to Withstand the Strike: 8 [Dame Beatrice’s WithIn] + 3 [Dame Beatrice’s Lucky Break bonus] = 11. She rolls a 6: an excellent WithIn roll!

Note two things: (1) Beatrice had 3 spare APs before the Lucky Break, but now has only 2; (2) Actions Beatrice took this round before declaring her Lucky Break do not get the benefit of her +3 Lucky Break bonus. For example, Bea- trice still fails the Strike roll she attempted earlier in the round.
\end{mdframed}

A Lucky Break is not considered an action. So characters do not incur an Action Penalty for giving themselves a Lucky Break (on Action Penalties, see chapter two, Skills).

\begin{mdframed}
EXAMPLE

Sir Derek Cape declares two Strikes this round. Thus, he incurs an Action Penalty of -1 for the duration of the round.

Sir Derek’s Strike is normally 10, but right now it’s 7 be- cause of a -3 Injury Penalty. So Sir Derek needs to roll 6 or less to make his Strikes: 10 [Sir Derek’s Strike] - 3 [Injury Penalty] - 1 [Action Penalty] = 6. Sir Derek decides to give himself a Lucky Break. He spends 1 AP and adds his Luck stat (3) to all subsequent Skill rolls this round.

Sir Derek now needs to roll 9 or less to Strike successfully: 6 [Sir Derek’s modified Strike; see above] + 3 [Lucky Break Bonus] = 9.

Giving himself a Lucky Break did not increase Sir Derek’s Action Penalty; Lucky Breaks are not considered actions.
\end{mdframed}

It is possible to get double, triple, or even greater Lucky Breaks by spending two or more APs.

\begin{mdframed}
EXAMPLE

Dame Beatrice is in a jam. She has 2 APs to spare. She decides to use them both to get a double Lucky Break. Dame Beatrice’s Luck stat is 3. By using 2 APs instead of one, she gets to add 6 to all Skill rolls for the duration of the round: 3 [Beatrice’s Luck] x 2 [number of APs spent] = 6.
\end{mdframed}

Keep in mind, however, that no matter how big your Lucky Break bonus is, a roll of 12 always fails.

COMPOSITES

Composites are derived from a combination, or ‘composite’, of two Attributes. There are three Composites: the Combat Composite, the Priestcraft Composite and the Witchcraft Composite.

The Combat Composite is a combination of the Vigour and Agility Attributes. The Priestcraft Composite is a combination of the Stamina and Intuition Attributes. The Witchcraft Composite is a combination of the Intellect and Luck Attributes.

This may sound confusing but it’s very simple once you see it in action.

COMPOSITES AND SKILLS

Like Attributes, Composites have associated Skills (called Composite Skills). The Combat Composite produces Combat Skills, the Priestcraft Composite gives Priestcraft Skills, and the Witchcraft Compos- ite is the basis for Witchcraft Skills.

Composites serve as the starting-point for Compos- ite Skill stats in the same way as Attributes do for normal Skill stats. When you create your character, her Composite Skill stats start out equal to the Composite stat from which they derive. You then improve Composite Skill stats with Advancement Points, just like normal Skill stats. Just as you can never improve your Attribute stats, you can also never improve your Composite stats.

\begin{mdframed}
EXAMPLE

You have created a new Priest character with a Priestcraft Composite of 3. So all your Priest’s Priest Skill stats start out as 3. You can use Advancement Points to improve those stats. But your character’s Priestcraft Composite stat will never change.
\end{mdframed}

THE COMBAT COMPOSITE
The actions used in Combat—running, jumping, lunging, swinging, aiming, and so on—test a character’s strength and nimbleness. In DR terms, Com- bat tests a character’s Vigour and Agility. The Skills that Combat requires cannot be said to be exclusively Vigour- or Agility-based. Rather, a good sword attack or a successful parry of an opponent’s blow relies on both Attributes. That’s why Combat Skills are not founded solely on either the Vigour or Agility Attribute, but are based instead on the Combat Composite.

There are nine Combat Skills: Strike, Missile Strike, Feint, Disarm, Brawling, Parry, Block, Dodge and Movement. They are explained in chapter seven, Combat.

Every character needs to work out her Combat Composite. Unlike Priestcraft and Witchcraft, which only some characters dabble in, all charac- ters will need to know how to attack and defend themselves.

THE PRIESTCRAFT COMPOSITE

The Priestcraft Composite combines the Stamina and Intuition Attributes. A Priest’s Stamina is signifi- cant because Priests channel divine energy through their bodies when using Priestcraft Skills. This can be a wearisome experience for the body. It is also wearisome on a Priest’s soul, which is why Intuition is the second part of the Priestcraft Composite: it serves here as a measure of the Priest’s spiritual strength.

There are eleven Priestcraft Skills. They are: Channel, Bless, Consecrate, Curse, Defile, Heal, Smite, Wrath, Prophesy, Resurrect, and Work Miracle. They are explained in chapter nine, Priestcraft.

Not all characters use Priestcraft Skills. Indeed, most characters do not. Unless you’re planning to use Priestcraft Skills, there is no need for you to work out your character’s Priestcraft Composite.

THE WITCHCRAFT COMPOSITE

The two parts of the Witchcraft Composite are Intel- lect and Luck. A Witch’s Intellect is significant be- cause a Witch’s grasp of the arcane knowledge that makes spellcasting possible is very demanding on a Witch’s mind, particularly his memory and ability to learn. A Witch’s Luck is important because the supernatural powers invoked by Witches are often random and unpredictable; a character with good fortune is more likely to be successful in controlling them.

There are eight Witchcraft Skills. They are: Al- chemy; Arcana; Conjuring; Enchantment; Hex; Illu- sion; Sorcery; and Summoning. See chapter ten, Witches, Magic and Spellcasting.

Like Priestcraft, most players do not need to work out their character’s Witchcraft Composite. Only do so if you are creating a spellcasting character.

FAVOURABLE ROUNDING

Composite stats are determined by taking the aver- age of the associated Attribute stats. For instance, if your character has a Vigour stat of 4 and an Agility stat of 2, her Combat Composite stat will be 3. But as often as not, the average will not be a whole number (like 2) but a fraction or decimal (like 2.5). For example, if your character has a Vigour stat of 4 and an Agility stat of 3, the average is 3.5. But you can’t have a stat of 3.5, because you can’t roll decimal numbers on a twelve-sided die. So frac- tions in DR must always be rounded up or down.

The Favourable Rounding rule determines how your character rounds fractions: either in her favour (which usually means up) or against her (which usually means down). Characters who enjoy Fa- vourable Rounding always round numbers in the way that is most favourable to them. In the case of Composite stats, for instance, characters with Fa- vourable Rounding round up. So a character with an average of 3.5 between her Vigour and Agility stats gets a Combat Composite of 4. Characters who do not have Favourable Rounding always round numbers in the way that is least favourable to them. So, in the example above, 3.5 would be rounded down to give a Combat Composite of 3.

How, then, do characters get Favourable Rounding? Whether or not a character enjoys Favourable Rounding is determined when you create your character. There are only two ways to get it. First, your character can get Favourable Rounding by rolling a 1 on any of the Character Generation Ta- bles. Second, your character automatically gets Favourable Rounding if you roll poorly when deter- mining your Attribute stats. A poor roll is one which results in 5 or fewer Attribute Points to divide be- tween your six Attributes.

For more on Favourable Rounding and character creation, see chapter four, Characters.

\end{document}
